\documentclass[11pt,a4paper]{moderncv}

% moderncv themes
\moderncvstyle{classic}      % style options are 'casual' (default), 'classic', 'oldstyle' and 'banking'
\moderncvcolor{black}         % color options are 'blue' (default), 'orange', 'green', 'red', 'purple', 'grey' and 'black'
\usepackage[utf8]{inputenc}  % replace by the encoding you are using

% adjust the page margins
\usepackage[scale=0.85]{geometry}

% A package that works with modern CV package
% https://ctan.javinator9889.com/macros/latex/contrib/moderntimeline/moderntimeline.pdf
\usepackage{moderntimeline}
% Set the scale.
\tlmaxdates{2013}{2024}
% Set the line width.
% This automatically sets the space under the top label to be 1pt more
\tlwidth{0.8ex}
% Set the labels text size
\tltext{\tiny}

% personal data
\firstname{Louis}
\lastname{Ledoux}
\title{Curriculum Vitae}
\born{26 February 1995}
\address{}{Barcelona, 08014}{Spain}
\phone[mobile]{+33~7~70~49~11~98}
\email{i.flledoux@gmail.com}
\homepage{bynaryman.github.io}
\social[linkedin]{ledoux-louis}
\social[twitter]{@L0u1s73doux}
\social[github]{Bynaryman}
\social[orcid]{0009-0006-3643-2157}
\social[googlescholar]{JWBIqG8AAAAJ}

\photo[90pt][0.4pt]{louis2.jpg}
%\extrainfo{Something to say}
\quote{``Post Hoc Ergo Propter Hoc''}

\nopagenumbers{}

\begin{document}
\makecvtitle

%\section{Research Interests}
%\textbf{Computer Architecture}(Floating-Point Unit, Systolic Arrays, Matrix-Matrix Multiply (MMM) units, GPUs, FPGAs),
%\textbf{Computer Arithmetic} (Number Representations, Application-Specific circuits, FloPoCo, Posit \& IEEE754, Kulisch Accumulator, accuracy and energy budgeting),
%\textbf{High Performance Computing} (BLAS, Dense \& Sparse GEMMs, Heterogeneous Workloads, Numerical Analysis, Supercomputing).

\section{Research Interests}

\textbf{Computer Architecture:} Floating-Point Units, Systolic Arrays, Matrix-Matrix Multiply (MMM) Units, GPUs, FPGAs

\textbf{Computer Arithmetic:} Number Representations, Application-Specific Circuits, FloPoCo, Posit \& IEEE754 Standards, Kulisch Accumulator, Accuracy and Energy Budgeting

\textbf{High Performance Computing:} BLAS, Dense \& Sparse GEMMs, Heterogeneous Workloads, Numerical Analysis, Supercomputing


\section{Education}
\tlcventry{2018}{0}{PhD Student}{\href{https://www.upc.edu/ca}{Universitat Politècnica de Catalunya (UPC)}}{Barcelona}{}
{
Supervisor: Marc Casas Guix. Thesis title: ``Floating-Point Arithmetic Paradigms for High-Performance Computing: Software Algorithms and Hardware Designs''.
%Co-working with UPC and BSC, I developped arithmetic tools for
}

\tlcventry{2015}{2018}{Engineering School}{\href{https://esir.univ-rennes.fr/en/esir-preparatory-cycle}{ESIR}}{Université de Rennes}{}
{
    Completed a three-year program culminating in an \textbf{Engineer diploma} certified by the CTI (Comité des Titres d'Ingénieurs) and a \textbf{Master's degree} (Magister) in Computer Science.
    %The curriculum focused on \textit{Computer Architecture, Parallel Programming Models, GPU/CUDA Programming, High-Performance Computing, VLSI Hardware Design, Compilation, Language Theory, etc...}
}

\tlcventry{2013}{2015}{Classe Préparatoire}{\href{https://esir.univ-rennes.fr/en/esir-preparatory-cycle}{CUPGE ESIR}}{Université de Rennes}{}
{
    Intensive program with a strong emphasis on Mathematics and Computer Sciences.
}


\section{Experience}

    \tlcventry{2018}{0}{Researcher}{Barcelona Supercomputing Center (BSC) - RoMoL/CAOS/SONAR}{Barcelona}{}{
        During The completion of my PhD I also worked as a researcher where I publisged peer-reviewed papers, travel to itnernational conferences.
        \begin{itemize}
            \item Conducted research on hardware acceleration and optimization using FPGA and ASIC technologies.
            \item Developed and implemented numerical algorithms tailored for FPGA-based computation.
            \item Collaborated with a multidisciplinary team to enhance the performance of large-scale simulations.
            \item \textbf{Keywords:} FPGA, Arithmetic, ASIC
        \end{itemize}
    }


    %\tlcventry{2017}{2018}{Hardware Engineer}{b$\left\langle \right\rangle$com}{Rennes}{}
    %{
    %    Coop student under a professionalization contract during my final year of study. Engaged in one year of R\&D focused on FPGA acceleration in the cloud, aiming to evaluate the feasibility of these novel solutions. Successfully integrated an IP (real-time SDR to HDR) to convert video from NVMe to NVMe through cloud FPGA. Developed the IP integration using HDLs, tweaked PCI-e drivers (EDMA and XDMA), and scheduled user processes asynchronously with OpenCL to overlap reads and writes, maximizing PCIe bandwidth to approximately 15.8 GB/s.
    %    \textbf{Keywords:} SDAccel, OpenCL, Xilinx FPGA (Ultrascale VU9P), VHDL, SystemVerilog, AWS F1 instances, Linux driver, Linux kernel, PCIe, C++1x, C, DMA, AaaS (Acceleration as a Service), FaaS (FPGA as a Service), Virtual Machine, Docker, PCI-e virtual functions.
    %}

    \tlcventry{2017}{2018}{Hardware Engineer}{b$\left\langle \right\rangle$com}{Rennes}{}
    {
        Coop student under a professionalization contract during my final year of study. Engaged in one year of R\&D focused on FPGA acceleration in the cloud, aiming to evaluate the feasibility of these novel solutions.
        \begin{itemize}
            \item Successfully integrated an IP (real-time SDR to HDR) to convert video from NVMe to NVMe through cloud FPGA.
            \item Developed the IP integration using HDLs and tweaked PCI-e drivers (EDMA and XDMA).
            \item Scheduled user processes asynchronously with OpenCL to overlap reads and writes, successfully saturating PCIe bandwidth ($\approx$15.8 GB/s).
            \item \color{blue} Traveled to https://www.xilinx.com/video/events/xdf-frankfurt-2018-keynote.htm https://www.xilinx.com/video/events/xdf-frankfurt-2018-keynote.htmll
        \end{itemize}
        \textbf{Keywords:} SDAccel, OpenCL, Xilinx FPGA (Ultrascale VU9P), VHDL, SystemVerilog, AWS F1 instances, Linux driver, Linux kernel, PCIe, C++1x, C, DMA, AaaS (Acceleration as a Service), FaaS (FPGA as a Service), Virtual Machine, Docker, PCI-e virtual functions.
    }


    \tldatelabelcventry{2017}{July 2017}{Back End Developer}{WaryMe}{Rennes}{}
    {
        Summer internship focused on developing the entire back end of a people security application.
        \begin{itemize}
            \item Designed and implemented backend services and APIs.
            \item Ensured secure data transmission using HTTPS and Let's Encrypt.
            \item Deployed and managed the application on AWS with PM2 and Nginx.
            \item Automated deployment processes with Jenkins.
            \item Collaborated with front-end developers using Angular and TypeScript.
        \end{itemize}
        \textbf{Keywords:} Node.js, SQLite3, Git, C++, C, Angular, TypeScript, HTTPS, Let's Encrypt, PM2, Nginx, AWS, Jenkins.
    }


    \tldatelabelcventry{2016}{July 2016}{Back End Developer}{ASKIA}{Paris 10\`eme}{}
    {
        During this summer internship, I developed an automated CLI tool for publishing surveys on popular platforms like GitHub and Zendesk. 
        \textbf{Key Concepts:} Node.js, HTTP(S), REST API, Git, Test-Driven Development (TDD), Event/Asynchronous Programming, Jasmine Framework, Mocks, Stubs.
    }


    \tldatelabelcventry{2014}{July 2014}{Electronics Technician}{Radio Electronique Rennaise (R.E.R)}{Rennes}{}{
	    Summer Internship...
        \begin{itemize}
            \item
            \item \textbf{Keywords:}
        \end{itemize}
    }

\section{Peer-reviewed Conference Papers}
\cvline{[LC23a]}{\textbf{L. Ledoux} and M. Casas, “An Open-Source Framework for Efficient Numerically-Tailored Computations,” in \textit{\href{https://fpl2023.org}{2023 33rd International Conference on Field-Programmable Logic and Applications (FPL)}}, Sep. 2023, pp. 19–26, Gothenburg, Sweden. \textit{Available: \href{https://doi.org/10.1109/FPL60245.2023.00011}{doi: 10.1109/FPL60245.2023.00011}, \href{https://arxiv.org/pdf/2406.02579}{arXiv:2406.02579}, \href{https://hal.science/hal-04277512}{HAL:04277512}}}
\cvline{[LC22]}{\textbf{L. Ledoux} and M. Casas, “A Generator of Numerically-Tailored and High-Throughput Accelerators for Batched GEMMs,” in \textit{\href{https://www.fccm.org}{2022 IEEE 30th Annual International Symposium on Field-Programmable Custom Computing Machines (FCCM)}}, May 2022, pp. 1–10, New York, USA. \textit{Available: \href{https://doi.org/10.1109/FCCM53951.2022.9786164}{doi: 10.1109/FCCM53951.2022.9786164}, \href{https://hal.science/hal-04103774}{HAL:04103774}}}

\section{Poster Presentations}
\cvline{[LC24b]}{\textbf{L. Ledoux} and M. Casas, “LLMMMM: Large Language Models Matrix-Matrix Multiplications Characterization on Open Silicon” in \textit{\href{https://www.bsc.es/education/predoctoral-phd/doctoral-symposium/11th-international-bsc-severo-ochoa-doctoral-symposium-2024}{2024 11th BSCSymposium}}, May 2024, Barcelona, Spain. \textit{Available: \href{https://hal.science/hal-04592229}{HAL:04592229}}}
\cvline{[LC24a]}{\textbf{L. Ledoux} and M. Casas, “The Grafted Superset Approach: Bridging Python to Silicon with Asynchronous Compilation and Beyond” in \textit{\href{https://osda.ws/}{2024 4th Workshop on Open-Source Design Automation (OSDA)}}, hosted at the International Conference on Design, Automation and Test in Europe Conference (\href{https://www.date-conference.com/}{DATE}), March 25, 2024, at Palacio De Congresos Valencia (Valencia Conference Centre - VCC), Valencia, Spain. \textit{Available: \href{https://hal.science/hal-04587458}{HAL:04587458}}}
\cvline{[LC23b]}{\textbf{L. Ledoux} and M. Casas, “Open-Source GEMM Hardware Kernels Generator: Toward Numerically-Tailored Computations” in \textit{\href{https://www.bsc.es/education/predoctoral-phd/doctoral-symposium/10th-international-bsc-severo-ochoa-doctoral-symposium-2023}{2023 10th BSCSymposium}}, May 2023, Barcelona, Spain. \textit{Available: \href{https://arxiv.org/abs/2305.18328}{arXiv:2305.18328}, \href{https://hal.science/hal-04094835}{HAL:04094835}}}

\section{Invited Talks}
\cvline{[LC19]}{\textbf{L. Ledoux} and M. Casas, “Accelerating DL inference with (Open)CAPI and posit numbers,” in \textit{\href{https://events19.linuxfoundation.org/events/openpower-summit-eu-2019/}{OpenPOWER Summit 2019}}, Lyon, France: Linux Foundation, Oct. 2019. \textit{Available: \href{https://hal.science/hal-04094850}{HAL:04094850}}}

\section{Miscelaneous}
\cvitem{ACM Summerschool}{attended }
\cvitem{Yale Patt}{attended anniversayr + yearly class}
\cvitem{Xilinx AI PAris}{}

\section{Languages}
\cvlanguage{French}{Native}{}
\cvlanguage{Spanish}{Native}{(with an honest french accent)}
\cvlanguage{English}{Full Proficiency}{}

\section{Skills}
\cvcomputer{Scripting}{Python, bash, sh, Linux}{Dissemination}{\LaTeX{}, Matplotlib, Inkscape}
\cvcomputer{FPGA}{AMD, ALtera, VHDL, Verilog, SystemVerilog, FloPoCo, SDAccel, AWS f1, PCI-e}{GPU}{CUDA 8,9, OpenCL}
\cvcomputer{Programming}{C, C++, Java, Scala}{Versioning}{git, github,gitlab, svn, recurrent pull requests}

\end{document}
