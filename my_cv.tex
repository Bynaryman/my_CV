\documentclass[11pt,a4paper]{moderncv}

% moderncv themes
\moderncvstyle{classic}      % style options are 'casual' (default), 'classic', 'oldstyle' and 'banking'
\moderncvcolor{black}         % color options are 'blue' (default), 'orange', 'green', 'red', 'purple', 'grey' and 'black'
\usepackage[utf8]{inputenc}  % replace by the encoding you are using

% adjust the page margins
\usepackage[scale=0.85]{geometry}

% A package that works with modern CV package
% https://ctan.javinator9889.com/macros/latex/contrib/moderntimeline/moderntimeline.pdf
\usepackage{moderntimeline}
% Set the scale.
\tlmaxdates{2013}{2024}
% Set the line width.
% This automatically sets the space under the top label to be 1pt more
\tlwidth{0.8ex}
% Set the labels text size
\tltext{\tiny}

% personal data
\firstname{Louis}
\lastname{Ledoux}
\title{Curriculum Vitae}
\born{26 February 1995}
\address{}{Barcelona, 08014}{Spain}
\phone[mobile]{+33~7~70~49~11~98}
\email{i.flledoux@gmail.com}
\homepage{bynaryman.github.io}
\social[linkedin]{ledoux-louis}
\social[twitter]{@L0u1s73doux}
\social[github]{Bynaryman}
\social[orcid]{0009-0006-3643-2157}
\social[googlescholar]{JWBIqG8AAAAJ}

\photo[90pt][0.4pt]{louis2.jpg}
%\extrainfo{Something to say}
\quote{``Post Hoc Ergo Propter Hoc''}

\nopagenumbers{}

\begin{document}
\makecvtitle

\section{Research Interests}
\cvline{Computer Architecture}{Floating-Point Unit, Systolic Arrays, GPUs, FPGAs}
\cvline{Computer Arithmetic}{Number Representations, Application-Specific circuits, FloPoCo, Posit \& IEEE754, Kulisch Accumulator, accuracy and energy budgeting}
\cvline{High Performance Computing}{BLAS, Dense Matrix-Matrix Multiplications, Heterogeneous Workloads, Numerical Analysis, Supercomputing}

\section{Education}
\tlcventry{2018}{0}{PhD Student}{\href{https://www.upc.edu/ca}{Universitat Politècnica de Catalunya (UPC)}}{Barcelona}{}
{
Completing the PhD in Computer Arithmetic
\begin{itemize}
 \item ddd
 \item DDD
\end{itemize}
}
\tlcventry{2015}{2018}{Engineering School}{\href{https://esir.univ-rennes.fr/en/esir-preparatory-cycle}{ESIR}}{Universit\'e de Rennes}{}
{
Three years to obtain the engineer diploma certified by the CTI (Comit\'e des Titres d'Ing\'enieurs) and a Master (Magister) in computer sciences
\begin{itemize}
 \item ddd
 \item DDD
\end{itemize}
}
\tlcventry{2013}{2015}{Classe Pr\'eparatoire}{\href{https://esir.univ-rennes.fr/en/esir-preparatory-cycle}{CUPGE ESIR}}{Universit\'e de Rennes}{}
{
\begin{itemize}
 \item 6/23 first year
 \item 7/22 second year
\end{itemize}
}

\section{Experience}

    \tlcventry{2018}{0}{Researcher}{Barcelona Supercomputing Center (BSC)}{Barcelona}{}{
	    researching..
        \begin{itemize}
            \item
            \item \textbf{Keywords:} FPGA, Arith, ASIC
        \end{itemize}
    }

    \tlcventry{2017}{2018}{FPGA as a Service Engineer}{b$\left\langle \right\rangle$com}{Rennes}{}{
        \begin{itemize}
            \item Coop student known as professionalization contract in French for my last study's year. One year of R\&D about Acceleration on FPGA on the cloud.
            \item \textbf{Keywords:} SDAccel, OpenCL, Xilinx FPGA (ultrascale vu9p), VHDL, SystemVerilog, AWS F1 instances, linux driver, linux kernel, PCIe, C++1x, C, DMA, AaaS (Acceleration as a Service).
        \end{itemize}
    }

    \tldatelabelcventry{2017}{July 2017}{Back End Developer}{WaryMe}{Rennes}{}{
	    Summer Internship...
        \begin{itemize}
            \item Development of all the back end of a people security application.
            \item \textbf{Keywords:} NodeJS, sqlite3, git, c++, C, Angular, TypeScript, https, let's encrypt, pm2, nginx, AWS, jenkins.
        \end{itemize}
    }

    \tldatelabelcventry{2016}{July 2016}{Back End Developer}{ASKIA}{Paris 10\`eme}{}{
	    Summer Internship...
        \begin{itemize}
            \item Development of a CLI tool which allows automated publication of surveys on well known platforms like Github or Zendesk.
            \item \textbf{Keywords:} NodeJS, http, api rest, git, test driven development, asynchronous programming, jasmine framework, mocks.
        \end{itemize}
    }
    \tldatelabelcventry{2014}{July 2014}{Electronics Technician}{Radio Electronique Rennaise (R.E.R)}{Rennes}{}{
	    Summer Internship...
        \begin{itemize}
            \item
            \item \textbf{Keywords:}
        \end{itemize}
    }
\section{Peer-reviewed Conference Papers}
\cvline{[LC23a]}{\textbf{L. Ledoux} and M. Casas, “An Open-Source Framework for Efficient Numerically-Tailored Computations,” in 2023 33rd International Conference on Field-Programmable Logic and Applications (FPL), Sep. 2023, pp. 19–26. \textit{\url{doi: 10.1109/FPL60245.2023.00011}}}

\cvline{[LC22]}{\textbf{L. Ledoux} and M. Casas, “A Generator of Numerically-Tailored and High-Throughput Accelerators for Batched GEMMs,” in 2022 IEEE 30th Annual International Symposium on Field-Programmable Custom Computing Machines (FCCM), May 2022, pp. 1–10. \textit{\url{doi: 10.1109/FCCM53951.2022.9786164}}}


\section{Poster Presentations}
\cvline{[LC24b]}{\textbf{L. Ledoux} and M. Casas, “LLMMMM: Large Language Models Matrix-Matrix Multiplications Characterization on Open Silicon” in 2024 11th BSCSymposium, May 2024, \textit{Available online soon.}}

\cvline{[LC24a]}{\textbf{L. Ledoux} and M. Casas, “The Grafted Superset Approach: Bridging Python to Silicon with Asynchronous Compilation and Beyond” in 2024 4th Workshop on Open-Source Design Automation (OSDA), hosted at the International Conference on Design, Automation and Test in Europe Conference (DATE), March 25, 2024, at Palacio De Congresos Valencia (Valencia Conference Centre - VCC), Valencia, Spain. \textit{Available online soon.}}

\cvline{[LC23b]}{\textbf{L. Ledoux} and M. Casas, “Open-Source GEMM Hardware Kernels Generator: Toward Numerically-Tailored Computations” in 2023 10th BSCSymposium, May 2023, \textit{Available: \url{https://arxiv.org/abs/2305.18328}}}

\section{Invited Talks}
\cvline{[LC19]}{\textbf{L. Ledoux} and M. Casas, “Accelerating DL inference with (Open)CAPI and posit numbers,” in OpenPOWER summit 2019, Lyon, France: linux foundation, Oct. 2019. \textit{Available: \url{https://hal.science/hal-04094850}}}

\section{Languages}
\cvlanguage{French}{Native}{}
\cvlanguage{Spanish}{Native}{(with an honest french accent)}
\cvlanguage{English}{Full Proficiency}{}

\section{Skills}
\cvcomputer{category 1}{XX}{category 2}{YY}
\cvcomputer{category 1}{XX}{category 2}{YY}
\cvcomputer{category 1}{XX}{category 2}{YY}
\cvitem{Programming}{Java, Python, C++}
\cvitem{Languages}{English, Spanish}

\end{document}
